
\documentclass[12pt]{article}
\usepackage{lingmacros}
\usepackage{tree-dvips}
\begin{document}

\section{Method}
\subsection{Citation Diversity Statement}

Recent work in neuroscience and other fields has identified a bias in citation practices such that papers from women and other minorities are under-cited relative to the number of such papers in the field \cite{Dworkin2020.01.03.894378, maliniak2013gender, caplar2017quantitative, chakravartty2018communicationsowhite, YannikThiemKrisF.SealeyAmyE.FerrerAdrielM.Trott2018, dion2018gendered}. Here we sought to proactively consider choosing references that reflect the diversity of the field in thought, form of contribution, gender, and other factors. We used automatic classification of gender based on the first names of the first and last authors \cite{Dworkin2020.01.03.894378, zhou_dale_2020_3672110}, with possible combinations including man/man, man/woman, woman/man, and woman/woman. Excluding self-citations to the senior authors of our current paper, the references contain $A\%$ man/man, $B\%$ man/woman, $C\%$ woman/man, $D\%$ woman/woman, and $E\%$ unknown categorization. We look forward to future work that could help us to better understand how to support equitable practices in science.

\newpage
\bibliographystyle{ieeetr}
\bibliography{./bibfile.bib}

\end{document}

